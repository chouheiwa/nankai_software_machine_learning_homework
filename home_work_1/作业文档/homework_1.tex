%! Author = chouheiwa
%! Date = 2022/10/21

% Preamble
\documentclass[UTF8]{article} %article 文档
\usepackage{ctex}  %使用宏包(为了能够显示汉字)
\usepackage{hyperref}
\usepackage{graphicx}
\usepackage{geometry}
\geometry{a4paper,scale=0.8}
\usepackage{listings}
\usepackage{color}
\definecolor{dkgreen}{rgb}{0,0.6,0}
\definecolor{gray}{rgb}{0.5,0.5,0.5}
\definecolor{mauve}{rgb}{0.58,0,0.82}
\lstset{frame=tb,
  language=Python,
  aboveskip=3mm,
  belowskip=3mm,
  showstringspaces=false,
  columns=flexible,
  basicstyle={\small\ttfamily},
  numbers=left,%设置行号位置none不显示行号
  %numberstyle=\tiny\courier, %设置行号大小
  numberstyle=\tiny\color{gray},
  keywordstyle=\color{blue},
  commentstyle=\color{dkgreen},
  stringstyle=\color{mauve},
  breaklines=true,
  breakatwhitespace=true,
  escapeinside=``,%逃逸字符(1左面的键),用于显示中文例如在代码中`中文...`
  tabsize=4,
  extendedchars=false %解决代码跨页时,章节标题,页眉等汉字不显示的问题
}

% Packages
\usepackage{amsmath}

% Document
\title{机器学习课程作业1}
\author{chouheiwa}
\date{2022/10/21}
\linespread{1.5}
\begin{document}
    \maketitle


    \section{问题一}
    这里无法避免使用到矩阵的算法,以及绘图的算法,所以代码中会额外需要使用到numpy(用以进行数学运算)和matplotlib(用以生成和绘制相关散点图及分界线图)这两个库。

    \subsection{第一小问}
    这里已知要生成N = 1000个二维样本的数据集$X_1$和$X_2$。又知晓样本来自于三个不同的正态分布,其分布的均值矢量分别为
    \[
        m_1 = [1, 1]^T\\m_2 = [4, 4]^T\\m_3 = [8, 1]^T
    \]
    协方差为
    \[
        \Sigma_1 = \Sigma_2 = \Sigma_3 = \begin{bmatrix}
                                             2 & 0 \\ 0 & 2
        \end{bmatrix}
    \]

    因此,我们可以先生成三个正态分布的随机数,然后再将其乘以协方差矩阵,最后再加上均值矢量,即可得到三个正态分布的样本。这里我们使用 numpy 的 random.multivariate\_normal 函数来生成对应数据集。

    生成数据集功能使用了名为GenerateData作为根据给定参数自动生成相关数据集的功能的类,其代码于文件\href{run:generate_data.py}{generate\_data.py}中。

    绘制数据图像功能代码于文件\href{run:plot_image.py}{plot\_image.py}中。

    main.py函数中会调用上述部分代码,生成数据集并绘制对应数据集的散点图图像。示例结果如下图所示。

\end{document}